\documentclass{article}
\usepackage{graphicx}
\usepackage{amsmath}
\usepackage{xcolor}
\usepackage[hidelinks]{hyperref}
\title{FIN 302 Project Overview\\
Tolling Contract and FTR Valuation }
\date{Fall 2018}
\begin{document}
\maketitle
\noindent In this real-world\footnote{I have done this analysis for deregulated utilities (notably NextEra Energy) and hedge funds.} project you will either be assigned to value a tolling contract on an existing power plant on the PJM interconnection, or a Financial Transmission Right (FTR) contract connecting two nodes in PJM.  Through the valuation you will:
\begin{enumerate}
\item apply `discounted cash flow' analysis.
% \item apply the `real options' approach to valuation.
% \item explain why (1) and (2) give you different answers.
\item encounter and solve real world capital budgeting and valuation problems, e.g. how will you forecast electricity prices?
\item learn a bit about deregulated wholesale electricity markets.  
% \item (possibly) try different option pricing models and explain the difference between them.
\end{enumerate}
Note that in analyses such as these, the calculations are often the easy part.  What you may find more difficult is choosing which assumptions you make, as well as overall organization and presentation.  \\
\\
{\it Assumptions}\\
In any real-world analysis you have a trade-off between fewer assumptions which lead to harder calculations and more time spent, and more assumptions which leads to greater pricing error.  Thus you will have to make that tradeoff with this project.  You are free to make any assumption you wish, but you must be ready to defend this assumption (e.g. `the cost savings in terms of time saved from the assumption is worth more than the small pricing error the assumption induces').      \\
\\
{\it Electricity Markets}\\
I assume none of you are familiar with deregulated electricity markets, so I advise each group to make use of the project discussion board.  Do not hesitate to ask questions.  By the end of the project you will know a fair amount about these markets---which is useful as more and more electricity markets are deregulated. 
\begin{center}
{\bf Some general notes}
\end{center}
\begin{enumerate}
\item This project will require heavy use of Excel, or other spreadsheet software (see the syllabus for free alternatives to Excel).  
\item When you value your tolling contract or FTR as an option you will most likely want to start with the Margrabe (1978) option pricing formula.  This formula prices the option to exchange one asset for another in the Black-Scholes (1973) framework---which means prices follow geometric Brownian motions.  The Margrabe formula is going to give you too high a price for your option, but since more correct formulas are harder to implement you may start with Margrabe.  If any group is interested, I will help you implement a more appropriate formula---Deng, Johnson, and Sogomonian (2001) or help you value the option via Monte Carlo simulation.
\item Electricity prices are highly seasonal and also vary according to a common pattern throughout the day.  They also can differ markedly by location.  So do not use your PnodeID's average price at 2pm in July 2013 to predict the average 2pm price for November 2013.  Use the average 2pm price for November 2012.
\end{enumerate}
\begin{center}
{\bf DATA}
\end{center}
Useful data can be found at the following sources.
\begin{enumerate}
\item You can get average monthly locational marginal prices\footnote{stated simply, this is the electricity price at each location of the grid (known by its unique PNodeID).} (LMP) by hour and by PnodeID \href{http://www.pjm.com/markets-and-operations/energy/real-time/monthlylmp.aspx}{\textcolor{green}{here}}.  There are three files for each month and you want the one ending in `rt', which stands for `real-time'. From these files you want the `TotalLMP' price by hour.
\item  Those of you valuing a power plant will need natural gas prices at the Henry Hub.  These daily prices can be found at the EIA website \href{http://www.eia.gov/dnav/ng/ng_pri_fut_s1_d.htm}{\textcolor{green}{here}}.
\item FTR auction results can be found \href{http://www.pjm.com/markets-and-operations/ftr/auction-user-info/historical-ftr-auction.aspx}{\textcolor{green}{here}} [you don't need this, it is just interesting background information].
\end{enumerate}
{\bf Using $SQL$ to get the data}\\
In case you would like to try pulling data from a $MySQL$ database, I will put the tables you need into my $MySQL$ database located at complete-markets.com.  If you want to use it you'll have to let me know so we can whitelist your IP address.  You'll also need to download a database client (MySQL Workbench from Oracle is full-featured).  Also, I am happy to help you write your SELECT queries.  Knowing $SQL$ is great to have on your resume, particularly if you are interested in careers at places like PNC Bank. 
\end{document}

%%% Local Variables:
%%% mode: latex
%%% TeX-master: t
%%% End:
