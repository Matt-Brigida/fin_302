\documentclass{article}
\usepackage{graphicx}
\usepackage{amsmath}
\usepackage{xcolor}
\usepackage[hidelinks]{hyperref}
\title{FIN 302 Project \\
Financial Transmission Right (FTR) Option Valuation \\
\vspace{15pt}
\large{{\bf Group 5}}\\
\large{}}
\date{Fall 2018}
\begin{document}
\maketitle
\noindent Value the following FTR option between PJM network nodes.  You will turn in an Excel file with your calculations. A well-organized spreadsheet should allow me to change some of your assumptions to see the effect on the FTR value.  Within the spreadsheet in a text box, you should supply a couple of paragraphs explaining your assumptions and methodology.  You should also supply your level of confidence in your estimates (you can do this in Monte Carlo fashion by varying the inputs and observing your range of FTR values).\\\\
\\
Contract Details:\\
\begin{enumerate}
\item Your contract is {\it the option} to transmit 35 MW per hour, for every on-peak hour from 6/1/2019 to 11/30/2019 (inclusive of start and end dates).
\item Note on-peak means 5x16, or Monday through Friday from the hour ending 0800 to the hour ending 2300.  
\item The FTR is for the transmission from (source) PnodeID 5022443 to (sink) PnodeID 51293.  Both points are on the PJM Interconnection. Note, you cannot transmit in the other direction.  
\item You estimate you'll have to pay an administration fee of 0.25\% per dollar paid to you from the FTR. 
\item For each day you can either transmit 35 MW per hour or nothing.  Therefore for each day you will transmit 16*35 = 560 MWh or nothing.
\item  In addition to the FTR option above, you will also have the right to transmit an additional 60 MW for each on-peak hour on one day between 6/1/2019 to 11/30/2019 inclusive.  Once you exercise this right on a particular day, the right ceases to exist.    
\end{enumerate} 
The structure of an FTR option in such that if over 1 hour the spot price of electricity at the sink (denote by $S_K$) is greater than the spot price of electricity at the source (denote by $S_E$) then you are paid (in cash) $S_K-S_E$ for every MW of your FTR.  Alternatively, if $S_E > S_K $ then you are paid \$0.  Importantly, you don't owe anything  in the latter scenario.  \\
\\
Given the above, the cash flow generated by each MW for which you own an FTR option is: $CF=max(S_K-S_E, \; 0)$ where $CF$ denotes cash flow.  Note, this is the payment from the FTR - you'll have to include other fees in your calculation.
\end{document}

%%% Local Variables:
%%% mode: latex
%%% TeX-master: t
%%% End:
